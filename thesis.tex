%%%%%%%%%%%%%%%%%%%%%%%%%%%%%%%%%%%%%%%%%%%%%%%%%%%%%%%%%%%%%%%%%%%%
%% I, the copyright holder of this work, release this work into the
%% public domain. This applies worldwide. In some countries this may
%% not be legally possible; if so: I grant anyone the right to use
%% this work for any purpose, without any conditions, unless such
%% conditions are required by law.
%%%%%%%%%%%%%%%%%%%%%%%%%%%%%%%%%%%%%%%%%%%%%%%%%%%%%%%%%%%%%%%%%%%%

\documentclass[
  digital,     %% The `digital` option enables the default options for the
               %% digital version of a document. Replace with `printed`
               %% to enable the default options for the printed version
               %% of a document.
%%  color,       %% Uncomment these lines (by removing the %% at the
%%               %% beginning) to use color in the printed version of your
%%               %% document
  oneside,     %% The `oneside` option enables one-sided typesetting,
               %% which is preferred if you are only going to submit a
               %% digital version of your thesis. Replace with `twoside`
               %% for double-sided typesetting if you are planning to
               %% also print your thesis. For double-sided typesetting,
               %% use at least 120 g/m² paper to prevent show-through.
  nosansbold,  %% The `nosansbold` option prevents the use of the
               %% sans-serif type face for bold text. Replace with
               %% `sansbold` to use sans-serif type face for bold text.
  nocolorbold, %% The `nocolorbold` option disables the usage of the
               %% blue color for bold text, instead using black. Replace
               %% with `colorbold` to use blue for bold text.
  lof,         %% The `lof` option prints the List of Figures. Replace
               %% with `nolof` to hide the List of Figures.
  lot,         %% The `lot` option prints the List of Tables. Replace
               %% with `nolot` to hide the List of Tables.
]{fithesis4}
%% The following section sets up the locales used in the thesis.
\usepackage[resetfonts]{cmap} %% We need to load the T2A font encoding
\usepackage[
  main=english, %% By using `czech` or `slovak` as the main locale
                %% instead of `english`, you can typeset the thesis
                %% in either Czech or Slovak, respectively.
  english, czech %% The additional keys allow
]{babel}        %% foreign texts to be typeset as follows:
%%

\newcommand{\TODO}[1]{\textcolor{red}{\textit{#1}}}

%% The following section sets up the metadata of the thesis.
\thesissetup{
    date        = \the\year/\the\month/\the\day,
    university  = mu,
    faculty     = fi,
    type        = bc,
    department  = \TODO{Department of Machine Learning and Data Processing},
    author      = Petr Bém,
    gender      = m,
    advisor     = {RNDr. Jan Mrázek},
    title       = {Firmware for RoFICoM},
    TeXtitle    = {Firmware for \acrshort{roficom}},
    keywords    = {\TODO{modular, robots, RoFI, firmware}},
    TeXkeywords = {\TODO{keyword1, keyword2, \ldots}},
    abstract    = {%
      This is the abstract of my thesis, which can

      span multiple paragraphs.
    },
    thanks      = {%
      These are the acknowledgments for my thesis, which can

      span multiple paragraphs.
    },
    bib         = citations.bib,
    %% Remove the following line to use the JVS 2018 faculty logo.
    facultyLogo = fithesis-fi,
}
\usepackage{makeidx}      %% The `makeidx` package contains
\makeindex                %% helper commands for index typesetting.
\usepackage{hyperref}
\usepackage[acronym]{glossaries}          %% The `glossaries` package
\renewcommand*\glspostdescription{\hfill} %% contains helper commands
\loadglsentries{terms-abbrs.tex}  %% for typesetting glossaries
\makenoidxglossaries                      %% and lists of abbreviations.
%% These additional packages are used within the document:
\usepackage{paralist} %% Compact list environments
\usepackage{amsmath}  %% Mathematics
\usepackage{amsthm}
\usepackage{amsfonts}
\usepackage{siunitx} %% SI units
\usepackage{url}      %% Hyperlinks
\usepackage{markdown} %% Lightweight markup
\usepackage{listings} %% Source code highlighting
\lstset{
  basicstyle      = \ttfamily,
  identifierstyle = \color{black},
  keywordstyle    = \color{blue},
  keywordstyle    = {[2]\color{cyan}},
  keywordstyle    = {[3]\color{olive}},
  stringstyle     = \color{teal},
  commentstyle    = \itshape\color{magenta},
  breaklines      = true,
  language        = c++,
}
\usepackage{floatrow} %% Putting captions above tables
\floatsetup[table]{capposition=top}
\usepackage[babel]{csquotes} %% Context-sensitive quotation marks
\usepackage{todonotes} % colorful TODO notes, REMOVE
\usepackage{xcolor}

\begin{document}
%% Uncomment the following lines (by removing the %% at the beginning)
%% and to print out List of Abbreviations and/or Glossary in your
%% document. Titles for these tables can be changed by replacing the
%% titles `Abbreviations` and `Glossary`, respectively.
\clearpage
\printnoidxglossary[title={Abbreviations}, type=\acronymtype]
\printnoidxglossary[title={Glossary}]

\chapter{Introduction}

\chapter{Preliminaries}
\TODO{
\begin{itemize}
    \item What is \acrshort{rofi} Platform
    \item What is \acrshort{roficom}
    \begin{itemize}
        \item Describe technicalities -- RAM: 36KB, Flash: 128KB, ...
    \end{itemize}
    \item Explain \acrshort{i2c}
\end{itemize}
}

\section[ RoFI ]{ \acrshort{rofi} }
The \acrshort{rofi} is a platform of distributed modular robots developed at Masaryk University, Faculty of Informatics.

\section[ RoFICoM ]{ \acrshort{roficom} }


\subsection{Technical details}

\section[Inter-Integrated Circuit communication protocol]{\acrlong{i2c} communication protocol}
\acrshort{i2c}, which stands for \acrlong{i2c}, is a protocol designed for communication between a controller (or more controllers) and multiple peripheral devices. The protocol requires only two lines: SDA (serial data) and SCL (serial clock). Both lines are open-drain with pull-up resistors, which means that the bus driver can drive the bus line low but cannot drive it high. This constrain prevents the collision  driving the line both high and low, which would damage the drivers.

The protocol's communication is done in 8-bit packets. The transaction starts with the start condition. After that, every packet is confirmed from the receiving side with one acknowledgment bit, and the transaction ends with the stop condition. Since more peripheral devices can be connected to the \acrshort{i2c} bus, the controller must first address the peripheral to communicate.

The address of a peripheral is either a 7-bit or 10-bit number, where the 10-bit address must be split into two packets with a specified format. The reason why the address is only 7-bit and not the full 8-bits of the packet is because the last bit indicates the transaction type. The transaction type is either read transaction (the bit is set to 0) or write transaction (the bit is set to 1). The read/write bit is the least significant bit of the packet, and the address in contained in the seven highest bits of the packet (i.e. the address is shifted left by one bit).After the address is sent and met with acknowledgment from the peripheral, the data begins to be transmitted. The protocol allows for any length of the data to be transmitted.

The start condition is when the controller drives the SDA line low before the SCL goes down. Stop condition is when the controller releases SDA (the line goes from low to high) \textbf{after} the SCL line changed to high. It is suggested not to change the SDA line while SCL is high to avoid signaling false stop conditions during data transmission. Note that the acknowledgment bit is negated, i.e., logical 0 on the bus line signals that the peripheral received the packet successfully, whereas logical 1 signals the opposite.

This protocol has several advanced topics, such as 10-bit addresses, repeated start conditions, and clock stretching, which are not needed to understand this thesis.

\TODO{Should I write about advanced topics? or at least 10-bit addresses?}

\chapter[ State of RoFICoM ]{ State of \acrshort{roficom} }
\TODO{
\begin{itemize}
    \item Project structure/architecture
    \item \texttt{control\_board}
    \item \texttt{stm32cxx}
    \item \texttt{rofi-hal}
\end{itemize}
}

\chapter[LiDAR]{ \acrshort{lidar} }
\TODO{
\begin{itemize}
    \item \TODO{Walk through \acrshort{lidar} features}
    \item \TODO{Communicating by \acrshort{i2c}}
    \item \TODO{\acrshort{lidar} protocol}
\end{itemize}
}

\acrshort{lidar}, which stands for \acrlong{lidar}, is a device that measures range. \acrshort{lidar} uses laser and time, in which the light travels to an object or a surface and bounces back to determine the distance. Thus, the results of this technique is very dependant on the ambient light in the measured environment. As such \acrshort{lidar} is not only used for measuring the distance from an object but also for \textbf{3-D scanning}. 3-D scanning is \TODO{collecting data about the real world environment} into a digital representation. Many industries use \acrshort{lidar}, one of the most known use cases might be in NASA's Mars helicopter Ingenuity \cite{garmin-lidar}.

\acrshort{roficom} uses \gls{vl53l1x} \acrshort{lidar} from STMicroelectronics. This specific \acrshort{lidar} has the following main features: (All features and description can be found in the part manual \cite{vl53l1x}.)

\begin{itemize}
    \item the ability to accurately measure the range between \qty{4}{\centi\metre} and \qty{4}{\metre},
    \item up to \qty{50}{\hertz} ranging frequency,
    \item single \qty{2.8}{\volt} power supply,
    \item \acrshort{i2c} communication interface with up to \qty{400}{\kilo\hertz} speed,
    \item shutdown and interrupt pins. (However, these pins are not necessary.)
\end{itemize}

Even though \acrshort{lidar} accurately measures the range between \qty{4}{\centi\metre} and \qty{4}{\metre}, it also measures distances below and above this range, although the measured distance does not have to be accurate. The lidar library, which \acrshort{roficom} uses, deals with inaccuracy by reporting the status of measured data, which is explained later on.

The shutdown pin is used to reset the \acrshort{lidar} by software. That being said, on the actual hardware, the pin is inverted, which results in behavior more similar to enable pin, i.e., the device is enabled when the bus with a shutdown pin is driven high and the device shuts down when the bus is driven low. This pin does not have to be controlled by software. However, in that case, it still has to be connected to the power supply through a pull-up connector, which was the case in the \acrshort{roficom} version 0.6.

On the other hand, the interrupt pin does not require to be connected. This pin signals that the new distance was measured and must be cleared before the next measurement. When the pin is not connected, it is never driven high, and thus the device is in always ready-to-measure state, which can result in missing measurements when the data is not received at the correct time.

The alternative to the interrupt signal is polling. Polling means that the firmware asks the \acrshort{lidar} periodically whether the measurement data are ready to be transmitted. However, if the period is too long, it is possible to miss some measured data. For example, if the inter-measurement timing (the time between two measurements) was \qty{100}{\milli\second} and the firmware's asking period was \qty{500}{\milli\second}, the firmware could miss four measurements from the \acrshort{lidar}. Polling was used with \acrshort{roficom} version 0.6 because the MCU did not have available pins for the interrupt pin. Whereas the \acrshort{roficom} version 1.0 is connected and uses the interrupt pin.

\section{ Distance modes }
\gls{vl53l1x} has 3 distance modes: short, medium and long. 

\section{ Communication }
\gls{vl53l1x} uses the \acrshort{i2c} communication protocol. However, there is a small but significant change in communication. \gls{vl53l1x} not only uses an \acrshort{i2c} address but also requires the address of its register, which is a 16-bit number and \textbf{must} be the first two data packets. The most significant byte of this index is sent first, followed by the least significant byte. Sending the index applies to both types of transactions (read and write). That means that in order to read data from the \acrshort{lidar}, the controller needs to make two transactions: first, a write transaction with register index, and second with the actual read. However, writing data to the \acrshort{lidar} makes only one write transaction because the index and data \textbf{must} be written together in one transaction.

\section{ \TODO{Protocol / API} }
Even though \gls{vl53l1x} has clearly defined read and write operations, there is no public specification of how to control the \acrshort{lidar}, i.e. how to initialize the \acrshort{lidar}, start measuring and request the measured data. STMicroelectronics provides and recommends using their software API, in a form of a software driver, to control the \acrshort{lidar}. The driver is a C library and is dual licensed under STMicroelectronics' proprietary license and under BSD 3-clause license, which is compatible with \acrshort{rofi}'s MIT license.

STMicroelectronics provides three driver versions: Full API (standard), Ultra Lite Driver (ULD) and Ultra Low Power (ULP). The full API or standard driver implements a plethora of functionality together with errors and error messages. The Ultra Lite strips down the API only to necessary functionality and calibration, which can be removed if the space is needed. The Ultra Low Power driver aims for the lowest power usage possible.

\todo{Where to mention used library -- here or in \ref{uld}}

The ultra lite driver was selected for the \acrshort{roficom} because the full API was too large for the firmware. The size of the full api driver's release build took up to \qty{19}{\kilo\byte} of the MCU's Flash. In comparison the release build of ultra lite driver takes only \qty{1}{\kilo\byte}. This resulted in missing errors and error messages that was previously used with the full API driver. The ULD only reports failure from the under-laying layer and we lose the error message. \todo{How to solve this error handling}.

\subsection{API of ULD}
Because the ULD API is platform independent, it requires the user to implement \acrshort{i2c} communication for the platform (the implementation of \acrshort{i2c} in \acrshort{roficom} is described in section \ref{lidar-i2c-impl}). The UM2510 \cite{um2510} manual explains how to use the ULD API and will be described in this section.

The flow of communication is following:

Wait for successful boot of the \acrshort{lidar}.

$\rightarrow$ Initialize the \acrshort{lidar}.

$\rightarrow$ Optional settings for the \acrshort{lidar}, such as changing the \acrshort{lidar} \acrshort{i2c} address, changing the distance mode, changing timings for measurements, \&c.

$\rightarrow$ Start ranging, i.e. measuring the data.

$\rightarrow$ Wait for the data to be ready -- either by polling or by interrupt.

$\rightarrow$ Receive the data from \acrshort{lidar}.

$\rightarrow$ Prepare for next ranging -- clear the interrupt flag.

$\rightarrow$ Either repeat from waiting for measurements or stop the measuring.

\todo{Insert API flow diagram}

\TODO{Describe distance modes or before.}
\TODO{Describe the timings or in lidar description.}
\TODO{Describe Result structure.}

\chapter{ Implementation - Design decisions }
This chapter walks over interesting design decisions and explains their reasoning.

\section[I2C driver]{\acrshort{i2c} driver}

\section{ Error handling }
\TODO{
\begin{itemize}
    \item \lstinline{atoms::Result} = exceptions are too heavy for embedded, \lstinline{std::expected} yet to come \texttt{c++23}
    \item the reasoning for given result types - \lstinline{atoms::Void}, \lstinline{std::string_view}
    \item \TODO{error handling in \acrshort{i2c} \& \acrshort{lidar} class}  
\end{itemize}
}

\section[BSP]{\acrlong{bsp}} \label{BSP}
\TODO{
\begin{itemize}
    \item \acrshort{bsp} = providing i2c functions to \acrshort{lidar} library + easier portability
    \item \acrshort{bsp} = global namespace \lstinline{Gpio::Pin} initialization is **UB**
\end{itemize}
} 

\acrlong{bsp}, or \acrshort{bsp} for short, is a layer of abstraction specifying  \TODO{minimal needed hardware requirements} for the project, and procedures for board initialization. For example needed pins and peripherals, such as \acrshort{i2c}. 

\TODO{main benefit: single point of gpio definitions, peripheral initialization, ... - enables for easier portability}

In \acrshort{roficom}, \acrshort{bsp} is implemented as two files: \verb|bsp.hpp| and \verb|bsp.cpp|. The \verb|bsp.hpp| header file contains all declarations of pins and peripheral drivers, which live in namespace \lstinline{bsp} and are exposed for use in \acrshort{roficom}. However, function \lstinline{bsp::setupBoard()} \textbf{must} be called before any pin or driver is accessed. The function \lstinline{bsp::setupBoard()} initialized the board, pins and drivers provided in the \lstinline{bsp} namespace. The header file provides declaration of function \lstinline{dbgInstance}, which is function that \acrshort{roficom} uses for debugging with a \TODO{ \texttt{uart} } driver. \verb|bsp.hpp| depends on external function \lstinline{SystemCoreClockUpdate()}, which is \TODO{platform-dependant function} that needs to be called after board clock has been setup. \TODO{In contrary} the \verb|bsp.cpp| defines and implements everything declared in \verb|bsp.hpp|. 

??

\verb|bsp.hpp| acts as an interface specifying what drivers and pins \acrshort{roficom} needs to function, however since those pins and drivers are stored as global variables in namespace \lstinline{bsp}, the interface should be kept as minimal as possible to save space in the global space. On the other hand \verb|bsp.cpp| acts as implementation of this interface and should mainly correctly setup the board, pins and drivers for usage, that means the \verb|bsp.cpp| can use pins and drivers that are not exposing as it needs to meet the interface specification.

To illustrate the main benefit of the \acrshort{bsp}, imagine scenario, where we need to change MCU on the board, thus changing pin layout. In this scenario we would only need to change the \verb|bsp.cpp| file with pin definitions (and if not already present implementation of drivers), but the \verb|bsp.hpp| and components built on top of that would be left unchanged and still work.

\subsection{Gpio::Pin initialization}
\lstinline{Gpio::Pin} initialization is the most notable and important change in \lstinline{bsp} that differs from all other uses in \acrshort{roficom}. \acrshort{roficom} uses the expression \lstinline{GpioA[ 9 ]} to specify used pins. \lstinline{GpioA}, \lstinline{GpioB} and \lstinline{GpioC} are global objects provided by the \lstinline{Gpio} driver. At the time of writing, the initialization order of global variables across multiple translation unit is not defined by the language specification. As a result \lstinline{bsp} cannot use \lstinline{GpioA}, ... for pin definitions and instead uses the \lstinline{Gpio::Pin} structure initialization, where first argument is the pin position and second argument is \TODO{the GPIO line defined by STMelectronics}.

\section{lidar?? hpp}

\subsection[LiDAR I2C implementation]{\acrshort{lidar} \acrshort{i2c} implementation} \label{lidar-i2c-impl}

\section{ \TODO{??} }
\TODO{
\begin{itemize}
    \item Hard fault = ram had too high static usage, minimizing the usage by lower the number of available buckets for packets solved this (memory::pool)
\end{itemize}
}

\TODO{At one point} during the development the firmware started hard faulting on the \acrshort{roficom}. Hard faults are exceptions that are typically caused by unrecoverable system errors. Such errors can be an execution of an Undefined instruction, an attempted load or store to an unaligned address, and many more, which are described in more detail in the ARM Devices Generic User Guide.  This resulted in corrupted stack frames during debugging. \TODO{} The Fix to this issue was reducing the size of \TODO{pre-allocated} packet buffer. The count of 1025B buffers was reduced from \TODO{20} to \TODO{5}. \TODO{This seemingly fixed the issue.}  

\section{ \TODO{CRPT} }
\begin{itemize}
    \item comparison CRPT vs inheritance approach
    \item inheritance = vtables can introduce waste of space
\end{itemize}

\section{ \TODO{Size optimization} }
\TODO{
\begin{itemize}
    \item Release build has over 107\% of MCU Flash
    \item `std::bind` -- `1,172B` vs Lambda `972B`
\end{itemize}
}

Function `lidarGet()` tries to receive measurement results from the \acrshort{lidar} and assumes that the device is successfully initialized. This function takes a reference to the `Lidar` class and variable to store in results. Because `lidarGet()` is a polling function it needs to be called indefinitely to constantly update the currently measured distance. Due to the nature of polling, it is needed to poll only when \acrshort{roficom} is idle, otherwise constant polling \TODO{without ...}

\subsection[Ultra Lite LiDAR library]{Ultra Lite \acrshort{lidar} library} \label{uld}

\TODO{
\begin{itemize}
    \item Use Low Power version due to saving space in Flash
    \item \TODO{Use `extern "C"` include guards}
\end{itemize}
}

\chapter[RoFICoM 1.0]{ \acrshort{roficom} 1.0 }

New \acrshort{roficom} revision arrived needing firmware updates to reflect changes in hardware.

\TODO{
Changes from version 0.6:
\begin{itemize}
    \item new MCU case with more pins: old = 29, new = 48 pins.
    \item due to a new case, several pins got changed
    \item \acrshort{lidar} changes: MCU pins for enable pin, interrupt pin
    \item Instead of 2 magnetic sensors as the limits of motor rotation, which didn't work quite enough, since the motor wasn't able to stop fast enough, version 2 has an array of 10 magnetic sensors to determine the speed and position of the motor in a more precise manner. 
\end{itemize}
}

On the previous revision, \acrshort{roficom}'s MCU (STM32G071GBU6) had only one last pin left unconnected. The old revision did not have connected \acrshort{lidar}'s enable pin, which meant there was no way to reset the \acrshort{lidar} from the firmware in case of a failure. The interrupt (IRQ) \acrshort{lidar} pin was also not connected in the old revision, but this was not an issue since it could have been solved by polling, as mentioned earlier in \TODO{Add a reference to the explanation of lidar usage}. The magnetic sensors for the motor's retraction and expansion limit could not detect and stop the movement of the motor fast enough, which damaged the \acrshort{roficom}'s case. Missing pins and a change in the motor's limit led to the change of MCU.

The newly chosen MCU (STM32G071CBUx) has \TODO{48} pins in total, which is a considerable upgrade from the \TODO{29} pins \TODO{that had the previous} MCU. Even though several pins changed \TODO{something is missing here}, reflecting the new hardware revision in the firmware was not as hard because of the usage of \acrshort{bsp}, as explained in section \ref{BSP}. Furthermore, \acrshort{lidar}'s enable and interrupt pins got connected to the MCU's GPIO pins.

As for the magnetic sensors, instead of two pieces, they got increased to an array of ten pieces along the motor's movement path to determine the motor's speed and position more precisely.

\chapter{Conclusion}


\appendix %% Start the appendices.
\chapter{An appendix}
Here you can insert the appendices of your thesis.

\end{document}
